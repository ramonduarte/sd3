%%%%%%%%%%%%%%%%%%%%%%%%%%%%%%%%%%%%%%%%%
% University/School Laboratory Report
% LaTeX Template
% Version 3.1 (25/3/14)
%
% This template has been downloaded from:
% http://www.LaTeXTemplates.com
%
% Original author:
% Linux and Unix Users Group at Virginia Tech Wiki 
% (https://vtluug.org/wiki/Example_LaTeX_chem_lab_report)
%
% License:
% CC BY-NC-SA 3.0 (http://creativecommons.org/licenses/by-nc-sa/3.0/)
%
%%%%%%%%%%%%%%%%%%%%%%%%%%%%%%%%%%%%%%%%%

%----------------------------------------------------------------------------------------
%	PACKAGES AND DOCUMENT CONFIGURATIONS
%----------------------------------------------------------------------------------------

\documentclass[a4paper,12pt]{article}

    % \usepackage[version=3]{mhchem} % Package for chemical equation typesetting
    % \usepackage{siunitx} % Provides the \SI{}{} and \si{} command for typesetting SI units
        \usepackage{graphicx} % Required for the inclusion of images
        \usepackage{natbib} % Required to change bibliography style to APA
        \usepackage{url}
    %     \usepackage{amsmath} % Required for some math elements 
        
    %     \setlength\parindent{0pt} % Removes all indentation from paragraphs
        
    % \renewcommand{\labelenumi}{\alph{enumi}.} % Make numbering in the enumerate environment by letter rather than number (e.g. section 6)
    

    %----------------------------------------------------------------------------------------
    %	DOCUMENT INFORMATION
    %----------------------------------------------------------------------------------------

    \title{Relatório do Trabalho Pr`atico 3} % Title
    \author{Ramon \textsc{Melo}} % Author name
    \date{\today} % Date for the report

    \begin{document}

        % \maketitle % Insert the title, author and date

        \begin{center}
            \begin{tabular}{l r}
                Data: & \today \\ % Date the experiment was performed
                Aluno: & Ramon Melo \\ % Partner names
                Professor: & Daniel Figueiredo % Instructor/supervisor
            \end{tabular}
        \end{center}

        % If you wish to include an abstract, uncomment the lines below
        % \begin{abstract}
        % Abstract text
        % \end{abstract}

        %----------------------------------------------------------------------------------------
        %	SECTION 1
        %----------------------------------------------------------------------------------------
        
        \section{Objetivo}
        
            Construir um sistema distribuído cujo mecanismo de ordenação total de eventos seja baseado no algoritmo \em{Totally Ordered Multicast}.
            
            % If you have more than one objective, uncomment the below:
            %\begin{description}
            %\item[First Objective] \hfill \\
            %Objective 1 text
            %\item[Second Objective] \hfill \\
            %Objective 2 text
            %\end{description}
            
            % \subsection{Definições}
            % \label{definitions}
            % \begin{description}
            % \item[Stoichiometry]
            % The relationship between the relative quantities of substances taking part in a reaction or forming a compound, typically a ratio of whole integers.
            % \item[Atomic mass]
            % The mass of an atom of a chemical element expressed in atomic mass units. It is approximately equivalent to the number of protons and neutrons in the atom (the mass number) or to the average number allowing for the relative abundances of different isotopes. 
            % \end{description}

        %----------------------------------------------------------------------------------------
        %	SECTION 2
        %----------------------------------------------------------------------------------------
        
        \section{Decisões de Projeto}
        
            \begin{tabular}{ll}
                Balance used & \#4\\
                Magnesium from sample bottle & \#1
            \end{tabular}
        
        %----------------------------------------------------------------------------------------
        %	SECTION 3
        %----------------------------------------------------------------------------------------
        
        \section{Implementação}
        
            \begin{tabular}{ll}
                Mass of magnesium metal & = sdsdsd
            \end{tabular}
            
        %----------------------------------------------------------------------------------------
        %	SECTION 4
        %----------------------------------------------------------------------------------------
        
        \section{Estudos de Caso}
        
        
        \begin{figure}[h]
            \begin{center}
                \includegraphics[width=0.65\textwidth]{placeholder_1} % Include the image placeholder.png
                \caption{Figure caption.}
            \end{center}
        \end{figure}
        
        %----------------------------------------------------------------------------------------
        %	SECTION 5
        %----------------------------------------------------------------------------------------
        
        \section{Considerações Finais}
        
        
        The most obvious source of experimental uncertainty is the limited precision of the balance. Other potential sources of experimental uncertainty are: the reaction might not be complete; if not enough time was allowed for total oxidation, less than complete oxidation of the magnesium might have, in part, reacted with nitrogen in the air (incorrect reaction); the magnesium oxide might have absorbed water from the air, and thus weigh ``too much." Because the result obtained is close to the accepted value it is possible that some of these experimental uncertainties have fortuitously cancelled one another.

        Random citation \cite{WEBSITE:2} embeddeed in text.

        %----------------------------------------------------------------------------------------
        %	BIBLIOGRAPHY
        %----------------------------------------------------------------------------------------
        
        \bibliographystyle{apalike}
        \bibliography{sample}
        
        %----------------------------------------------------------------------------------------


    \end{document}